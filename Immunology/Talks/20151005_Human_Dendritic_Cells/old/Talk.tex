\documentclass[a4paper]{article}
\usepackage{geometry}
\usepackage{cite}
\bibliographystyle{apalike}

\setlength{\parindent}{4em}
\setlength{\parskip}{1em}

\title{Human dendritic cells – from adult to fetus}
\author{Dr Naomi McGovern, SIgN, A*STAR, Singapore}
\date{Monday 5th October 2015}

\begin{document}
    \maketitle

    \section{Introduction}
The fetus presents a unique challenge in that the mother must support a semi-allogeneic organism within herself. Originally, it was thought that the placenta provided an impermeable barrier to the movement of cells between the mother and the fetus, and that immunity is achieved through isolation.

\subsection{However...}
However, many recent studies have shown the birectional movement of cells across the placenta, raising many questions about how immune tolerance is achieved. \cite{pmid14996783}

\subsection{An account of the new frontier of microchimerism}
Naturally acquired microchimerism has recently been investigated in autoimmune diseases, including scleroderma, thyroiditis, primary biliary cirrhosis, Sjögren syndrome, systemic lupus, dermatomyositis, and neonatal lupus. Iatrogenic chimerism has been investigated in transplantation and following blood transfusion. Considering findings of naturally acquired microchimerism along with iatrogenic microchimerism suggests microchimerism can have detrimental and/or beneficial effects in both settings. Recent identification of tissue-specific microchimerism either from naturally acquired or iatrogenic microchimerism (eg, cardiac myocytes) raises the possibility that microchimerism can be a target of autoimmunity or alternatively contribute to tissue repair. \cite{pmid19056990}

\subsection{Clinical evidence}
A study was performed in infants with SCID \cite{pmid11535520} to determine the presence of materal lymphocytes. It was found that many of these infants showed signs of GvHD (!)

\subsection{Immunological consequences}
\begin{itemize}
\item One theory is that all T cells are induced to become Treg cells (as proposed by \cite{pmid14996783}) such that all foreign antigens are tolerated. However, in utero transplantation tends to prevent excessive organ damage, provides long-term tolerance and is more successful with a fetal immunodeficiency.

\item The speaker gave an account of her journey in investigating the fetal immune system.
\end{itemize}

\section{What are dendritic cells?}
\begin{itemize}
\item Dendritic cells were discovered by Paul Langerhans in 1868. They were named for their starry shape.

\item Since then, we have learnt a lot about their functional role. Steinman received a Nobel Prize in 2011 for discovery of its role in adaptive immunity.

\end{itemize}

\section{What do we know about dendritic cells in the adult immune system?}

\subsection{Presentation}
They can present viral antigens to the CD4+ cells via MHC Class I molecules, as well as cross-presenting exogenous antigen via the same MHC Class I molecules. They can also induce the naive CD4+ cells to differentiate into Th1, Th2 Th17 or Treg cells depending upon the interleukins secreted.

\subsection{Characterisation of subtypes}
\begin{itemize}
\item cDC1s have been characterised by \cite{pmid22795876} to play a role in cross-presentation

\item cDC2 have been characterised by \cite{pmid20018619} to play a role in the polarisation of the CD4 response.

\item CD14+ cells have been thought to play a role in tolerogenic IL-10 release, \cite{pmid25200712} showed that they actually transcriptionally align with monocytes/macrophages. expressing DCSIGN but not CD26 upon extravasation from the blood.

\end{itemize}

\subsection{Other}
FETAL HAEMATOPOIESIS OCCURS IN THE LIVER UNTIL ~ 10 weeks THEN BONE MARROW AFTER THAT (with some overlap)

\section{DCs are abundant in fetal tissues}
Saw this by transcriptionally aligning adult and fetal dendritic cells to show mostly same genes expressed. Batch effect mitigated.

\subsection{What is the batch effect?}
Gene expression profiling (GEP) via microarray analysis is a widely used tool for assessing risk and other patient diagnostics in clinical settings. However, non-biological factors such as systematic changes in sample preparation, differences in scanners, and other potential batch effects are often unavoidable in long-term studies and meta-analysis. In order to reduce the impact of batch effects on microarray data, Johnson, Rabinovic, and Li developed ComBat for use when combining batches of gene expression microarray data. \cite{pmid25887219}

\section{Fetal DCs migrate to lymph nodes}
\begin{itemize}
\item Fetal DCs migrate in the human \cite{pmid22795876} and in the mouse \cite{pmid15308107}

\item From the skin to lymph \cite{pmid25200712}
\end{itemize}

\section{What immune responses do fetal DCs promote?}
\begin{itemize}
\item Fetal DCs promote Treg induction through FOXP3 and TGF$\beta$ - shown in graphs.

\item Differential gene expression between adults and fetus - Arg2 and CD71 high in fetus, low in adult.

\item CD71+ leads to immunosuppression \cite{pmid24196717}

\item Arg1 leads to immunosuppression, Arg2 closely related and seems to do the same.

\end{itemize}

\begin {table}[h]
\begin{center}
    \begin{tabular}{ | l | l | l |}
    \hline
 & Fetal T cells & Adult T cells \\ \hline
Ctl & + & + \\ \hline
+Fetal DCs & - & - \\
\hline
    \end{tabular}
\caption{Activity of T cells in presence and absence of fetal DCs. Normally, DCs lead to stimulation, not suppression}
\end{center}
\end {table}

\section{Conclusions}

\begin{itemize}
\item The fetus has a complete APC network by 13 weeks EGA.

\item These APCs are immunocompetent, actively promoting tolerance.

\item This has developmental, immunological and clinical consequences. Tolerance to maternal antigens is lifelong.

\item Arginase is released to convert L-arginine to ornithine, reducing its availability to NOS and helps to modulate immune responses. TNF$\alpha$ induces its release.
\end{itemize}

    \bibliographystyle{plain}
    \bibliography{Refs}
\end{document}

